\documentclass{article}
\begin{document}
\section{Pràctica III: Anàlisi semàntica i generació de codi intermedi}
Alumnes: Joaquim Picó Mora, Sergi Simón Balcells\\
Professora: Maria Teresa Alsinet Bernadó\\
Curs: 2020-2021
\subsection{Caràcteristiques bàsiques}
A continuació és mostra un llistat de les caràcteristiques principals
que s'han implementat en aquest llenguatge: 
\begin{itemize}
\item Tots els valors són constants 
\item S'ha implementat la definició de funcions (Veure a test/files/functions/normal\_function.chs) 
\item S'ha implementat la crida de funcions (Veure a test/files/functions/call\_func.chs) 
\item Adicionalment, s'ha implementat la currificació dels paràmetres de les funcions . (Veure a test/files/functions/currification.chs)
\item No hi ha constants globals, només funcions globals. Tot s'ha de definir dins alguna fuinció. 
\item Àmbits de programa: S'ha pensat com una llista(pila) que resideix en la mònada AlexState en la qual es van afegint Maps a mesura que es creen contextos diferents. 
\item Soporta els tipus bolea, enter, caràcter i real. 
\item Es poden crear arrays i arrays ndimensionals (Veure a test/files/arrays) 
\item S'han creat totes les expressions d'enters, dels reals, dels boleans i algunes per el tipus caràcter. (Veure a test/files/expresions) 
\item S'ha creat l'estructura d'assignació per a les funcions i per a data. 
\item S'ha creat l'estructura if/else (Veure a test/files/conditional/conditional.chs) 
\item S'ha creat l'estructura while (Veure a test/files/while/while.chs) 
\item S'ha creat l'estructura for (veure a test/files/for/for.chs)
\item S'ha creat l'estructura repeat/until (veure a test/files/repeat\_until/repeat\_until.chs)
\item Adicionalment s'ha creat l'estructura map (veure a test/files/map/map.chs)
\item Adicionalment s'ha implementat la definició de tipus mitjançant data. (Veure a test/files/data/data.chs) 
\item Juntament amb data s'ha implementat l'estructura case of. (Veure a test/files/case\_of/case\_of.chs)
\end{itemize}
\subsection{Three Address Code}
Es pot veure la sintàxis del còdi generat en els fitxers amb extensió
.tac de la carpeta test/files. En algun dels testos es proven casos d'error, i per tant en aquells no hi ha codi de tres adreces generat. S'ha de veure d'aquells en que no es prova un cas d'error.
\subsection{Taula de simbols}
En la mònada d'estat que ens dona Àlex i emmagatzemem dos Maps. El map
de definicions on s'hi guardaran les definicions de tipus que realitzem
mitjançant data. I el Map de valors, on s'hi guardara tota la resta de
definicions.
\end{document}

