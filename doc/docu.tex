\documentclass{article}
\begin{document}
\section{Pràctica III: Anàlisi semàntica i generació de codi intermedi}
Alumnes: Joaquim Picó Mora, Sergi Simón Balcells\\
Professora: Maria Teresa Alsinet Bernadó\\
Curs: 2020-2021
\subsection{Caràcteristiques bàsiques}
A continuació és mostra un llistat de les caràcteristiques principals
que s'han implementat en aquest llenguatge: 
\begin{itemize}
\item Tots els valors són constants 
\item S'ha implementat la definició de funcions (Veure atest/files/functions) 
\item S'ha implementat la crida de funcions 
\item Adicionalment, s'ha implementat la currificació dels paràmetres de les funcions 
\item No hi ha constants globals, només funcions globals. Tot és defineix dins una fuinció. 
\item Àmbits de programa: S'ha pensat com unallista(pila) que resideix en la mònada AlexState en la qual es vanafegint Maps a mesura que es creen contextos diferents. 
\item Soporta els tipus bolea, enter, caràcter i real. 
\item Es poden crear arrays i arrays ndimensionals (Veure a test/files/arrays) * S'han creat totes les expressions d'enters, dels reals, dels boleans i algunes per el tipus caràcter. (Veure a test/files/expresions) 
\item S'ha creat l'estructura d'assignació per a les funcions i per a data. 
\item S'ha creat l'estructura if/else (veure a test/files/conditional) 
\item S'ha creat l'estructura while (veure a test/files/while) 
\item S'ha creat l'estructura for (veure a test/files/for)
\item S'ha creat l'estructura repeat/until (veure a test/files/repeat\_until)
\item Adicionalment s'ha creat l'estructura map (veure a test/files/map)
\item Adicionalment s'ha implementat la definició de tipus mitjançant data. (Veure a test/files/data) 
\item Juntament amb data s'ha implementat l'estructura case of. (Veure a test/files/case\_of)
\end{itemize}
\subsection{Three Address Code}
Es pot veure la sintàxis del còdi generat en els fitxers amb extensió
.tac de la carpeta test/files. És la carpeta on hi ha tots els testos,
en els casos en que testejem errors no es jenera el codi de tres adreces
si no que es printa l'error. La sintàxi es pot observar en els fitxers
.tac que es generen a partir dels testos amb el còdi correcte.
\subsection{Taula de simbols}
En la mònada d'estat que ens dona Àlex i emmagatzemem dos Maps. El map
de definicions on s'hi guardaran les definicions de tipus que realitzem
mitjançant data. I el Map de valors, on s'hi guardara tota la resta de
definicions.
\end{document}

